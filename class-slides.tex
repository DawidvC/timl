% $Header$

\documentclass{beamer}

\usepackage{amsmath,amsfonts,amscd,amssymb,proof,MnSymbol}
\usepackage{mathpartir}
\usepackage{turnstile}% http://ctan.org/pkg/turnstile
\usepackage{adjustbox}% http://ctan.org/pkg/adjustbox
\usepackage{color}
\usepackage{amsthm}

\newtheorem{thm}{Theorem}
\newtheorem{lem}{Lemma}
\newtheorem{prop}{Proposition}

\renewcommand{\makehor}[4]
  {\ifthenelse{\equal{#1}{n}}{\hspace{#3}}{}
   \ifthenelse{\equal{#1}{s}}{\rule[-0.5#2]{#3}{#2}}{}
   \ifthenelse{\equal{#1}{d}}{\setlength{\lengthvar}{#2}
     \addtolength{\lengthvar}{0.5#4}
     \rule[-\lengthvar]{#3}{#2}
     \hspace{-#3}
     \rule[0.5#4]{#3}{#2}}{}
   \ifthenelse{\equal{#1}{t}}{\setlength{\lengthvar}{1.5#2}
     \addtolength{\lengthvar}{#4}
     \rule[-\lengthvar]{#3}{#2}
     \hspace{-#3}
     \rule[-0.5#2]{#3}{#2}
     \hspace{-#3}
     \setlength{\lengthvar}{0.5#2}
     \addtolength{\lengthvar}{#4}
     \rule[\lengthvar]{#3}{#2}}{}
   \ifthenelse{\equal{#1}{w}}{% New wavy $\sim$ definition
     \setbox0=\hbox{$\sim$}%
     \raisebox{-.6ex}{\hspace*{-.05ex}\adjustbox{width=#3,height=\height}{\clipbox{0.75 0 0 0}{\usebox0}}}}{}
  }

\newcommand{\thide}[1]{\left \langle #1 \right \rangle}
\newcommand{\typing}[4]{\turnstile{s}{s}{#4}{#3}{n}#1:#2}
\newcommand{\kinding}[2]{\turnstile{s}{s}{}{}{n}#1:#2}
\newcommand{\teq}[2]{#1\equiv#2}
\newcommand{\arrow}[4]{#1\xrightarrow[#3]{#2}#4}
\newcommand{\bottom}{\perp}
\newcommand{\symlet}{\mathsf{let\;}}
\newcommand{\symin}{\mathsf{\;in\;}}
\newcommand{\symletrec}{\mathsf{letrec\;}}
\newcommand{\symand}{\mathsf{\;and\;}}
\newcommand{\symmatch}{\mathsf{match}}
\newcommand{\FV}{\mathsf{FV}}
\newcommand{\symwith}{\mathsf{\;with\;}}
\newcommand{\symleft}{\mathsf{left}}
\newcommand{\symright}{\mathsf{right}}
\newcommand{\symmax}{\mathsf{max}}
\newcommand{\symSleft}{\mathsf{Sleft\;}}
\newcommand{\symSright}{\mathsf{Sright\;}}
\newcommand{\symfold}{\mathsf{fold\;}}
\newcommand{\symSfold}{\mathsf{Sfold\;}}
\newcommand{\symunfold}{\mathsf{unfold\;}}
\newcommand{\symSunfold}{\mathsf{Sunfold\;}}
\newcommand{\symhide}{\mathsf{hide\;}}
\newcommand{\symShide}{\mathsf{Shide\;}}
\newcommand{\symunhide}{\mathsf{unhide\;}}
\newcommand{\symSunhide}{\mathsf{Sunhide\;}}
\newcommand{\leO}{\preceq}
\newcommand{\sympair}{\mathsf{pair}}
\newcommand{\symtt}{\mathsf{tt}}
\newcommand{\symunit}{\mathsf{unit}}
\newcommand{\symlist}{\mathsf{list}}
\newcommand{\symnil}{\mathsf{nil}}
\newcommand{\symcons}{\mathsf{cons}}
\newcommand{\symfix}{\mathsf{fix}}
\newcommand{\symbool}{\mathsf{bool}}
\newcommand{\symtrue}{\mathsf{true}}
\newcommand{\symfalse}{\mathsf{false}}
\newcommand{\symmerge}{\mathsf{merge}}

%% \newcommand{\intro}[2]{#2^#1}
\newcommand{\intro}[2]{(#1 : #2)}
%% \newcommand{\intro}[2]{(#2 \mathsf{\;size\;} #1)}
%% \newcommand{\intro}[2]{\{#2 \mathsf{\;|\;} #1\}}

\newcommand{\symsum}{\mathsf{sum}}
\newcommand{\symfst}{\mathsf{fst}}
\newcommand{\symsnd}{\mathsf{snd}}
\newcommand{\symif}{\mathsf{if\;}}
\newcommand{\symthen}{\mathsf{\;then\;}}
\newcommand{\symelse}{\mathsf{\;else\;}}
\newcommand{\symSbool}{\mathsf{Sbool}}
\newcommand{\symuf}{\mathsf{uf}}
\newcommand{\symuh}{\mathsf{uh}}
\newcommand{\syml}{\mathsf{l}}
\newcommand{\symr}{\mathsf{r}}
\newcommand{\symf}{\mathsf{f}}
\newcommand{\syms}{\mathsf{s}}
\newcommand{\symmsort}{\mathsf{msort}}
\newcommand{\symSstat}{\mathsf{Sstat}}
\newcommand{\symsplit}{\mathsf{split}}
\newcommand{\symprod}{\mathsf{prod}}
\newcommand{\symStt}{\mathsf{Stt}}
\newcommand{\symSpair}{\mathsf{Spair}}
\newcommand{\symSlr}{\mathsf{Slr}}
%% \newcommand{\defeq}{\triangleq}
\newcommand\defeq{\mathrel{\overset{\makebox[0pt]{\mbox{\normalfont\tiny\sffamily def}}}{=}}}

% This file is a solution template for:

% - Talk at a conference/colloquium.
% - Talk length is about 20min.
% - Style is ornate.



% Copyright 2004 by Till Tantau <tantau@users.sourceforge.net>.
%
% In principle, this file can be redistributed and/or modified under
% the terms of the GNU Public License, version 2.
%
% However, this file is supposed to be a template to be modified
% for your own needs. For this reason, if you use this file as a
% template and not specifically distribute it as part of a another
% package/program, I grant the extra permission to freely copy and
% modify this file as you see fit and even to delete this copyright
% notice. 


\mode<presentation>
{
  \usetheme{Warsaw}
  % or ...

  \setbeamercovered{transparent}
  % or whatever (possibly just delete it)
}


\usepackage[english]{babel}
% or whatever

\usepackage[latin1]{inputenc}
% or whatever

\usepackage{times}
\usepackage[T1]{fontenc}
% Or whatever. Note that the encoding and the font should match. If T1
% does not look nice, try deleting the line with the fontenc.


\title
{$\lambda^{\forall,\omega,\mu}_c$: Complexity Recursive Types}

\author
{Peng Wang}
% - Give the names in the same order as the appear in the paper.
% - Use the \inst{?} command only if the authors have different
%   affiliation.

\institute[MIT CSAIL] % (optional, but mostly needed)
{
  MIT CSAIL
}
% - Use the \inst command only if there are several affiliations.
% - Keep it simple, no one is interested in your street address.

\date
{MIT 6.888, 2015}
% - Either use conference name or its abbreviation.
% - Not really informative to the audience, more for people (including
%   yourself) who are reading the slides online

\subject{Programming Languages}
% This is only inserted into the PDF information catalog. Can be left
% out. 



% If you have a file called "university-logo-filename.xxx", where xxx
% is a graphic format that can be processed by latex or pdflatex,
% resp., then you can add a logo as follows:

% \pgfdeclareimage[height=0.5cm]{university-logo}{university-logo-filename}
% \logo{\pgfuseimage{university-logo}}



% Delete this, if you do not want the table of contents to pop up at
% the beginning of each subsection:
%% \AtBeginSubsection[]
%% {
%%   \begin{frame}<beamer>{Outline}
%%     \tableofcontents[currentsection,currentsubsection]
%%   \end{frame}
%% }


% If you wish to uncover everything in a step-wise fashion, uncomment
% the following command: 

\beamerdefaultoverlayspecification{<+->}


\begin{document}

\begin{frame}
  \titlepage
\end{frame}

\begin{frame}{Outline}
  \tableofcontents
  % You might wish to add the option [pausesections]
\end{frame}


% Structuring a talk is a difficult task and the following structure
% may not be suitable. Here are some rules that apply for this
% solution: 

% - Exactly two or three sections (other than the summary).
% - At *most* three subsections per section.
% - Talk about 30s to 2min per frame. So there should be between about
%   15 and 30 frames, all told.

% - A conference audience is likely to know very little of what you
%   are going to talk about. So *simplify*!
% - In a 20min talk, getting the main ideas across is hard
%   enough. Leave out details, even if it means being less precise than
%   you think necessary.
% - If you omit details that are vital to the proof/implementation,
%   just say so once. Everybody will be happy with that.

\section{Example: Merge-Sort}

\begin{frame}{Example: Merge-Sort}

\begin{align*}
\symmsort &\defeq \lambda A. \lambda cmp.\;\symfix\;f(xs). \\
& \hspace{0.3in} \symmatch\;xs\symwith \\
& \hspace{.4in} |\; \symnil\Rightarrow xs \\
& \hspace{.4in} |\; \_::xs' \Rightarrow \symmatch\;xs'\symwith \\
& \hspace{1in} |\; \symnil\Rightarrow xs \\
& \hspace{1in} |\; \_ \Rightarrow \symmatch\; \symsplit\;xs \symwith \\
& \hspace{1.4in} |\; (ys, zs) \Rightarrow \symmerge\;cmp\;(f\;ys)\;(f\;zs) \\
\\
\symmsort &: \forall A.\arrow{(\arrow{A}{0}{1}{\arrow{A}{1}{1}{\symbool}})}{0}{1}{\arrow{\intro{s}{\symlist\;A}}{s*\log(s)}{s}{\symlist\;A}}
\end{align*}

\section{Recursive Types}

\end{frame}

\begin{frame}{Recursive Types}
\begin{align*}
\mathsf{nat} &\defeq \mu X.\; \symunit + X \\
\symlist &\defeq \Lambda A.\; \mu X.\; \symunit + A \times X \\
\mathsf{tree} &\defeq \Lambda A,B.\; \mu X.\; A + X \times B \times X
\end{align*}

\end{frame}

\section{Type Soundness}

\begin{frame}{Type Soundness}

\begin{thm}[Soundness w.r.t. Safety]
$$
\begin{array}{l}
\forall e,\tau. \\
\hspace{0.2in}  e:\tau \Rightarrow \\
\hspace{0.4in} \forall e'.\; e \leadsto^* e' \Rightarrow e'\in\mathsf{Val} \;\vee\; \exists e''. e' \leadsto e''.
\end{array}
$$
\end{thm}

\begin{thm}[Soundness w.r.t. Boundedness]
$$
\begin{array}{l}
\forall f,\tau_1,c,s,\tau_2. \\
\hspace{0.2in} f \downarrow \arrow{(x:\tau_1)}{c(x)}{s(x)}{\tau_2} \Rightarrow \\
\hspace{0.4in} \exists B. \;\forall v.\; v \downarrow \tau_1 \Rightarrow \\
\hspace{0.4in} \forall e',n.\; f\;v\leadsto^n e' \Rightarrow n\leq B\times c(|v|)
\end{array}
$$
\end{thm}

\end{frame}

\section{Proof Techniques}


\begin{frame}{Proof Techniques - First Attempt}

  \begin{thm}[Soundness]
    $$
     e:\tau \Rightarrow P(e)
    $$
  \end{thm}

  \begin{itemize}
  \item First Attempt: Proof by induction on typing derivation
  \item Problem: Application Case
  \begin{prop}[APP Case]
    $$
    P(e_1) \wedge P(e_2) \Rightarrow P(e_1\;e_2)
    $$
  \end{prop}
  \end{itemize}

\end{frame}


\begin{frame}{Logical Relation}

\begin{itemize}
\item Logical relation
  \begin{itemize}
  \item A systematic way to strengthen induction hypothesis
  \item Define a new interpretation of types : $\lsem\tau\rsem$
    \begin{itemize}
    \item $\lsem\tau\rsem$ is the set of ``good expressions'' of type $\tau$
    \end{itemize}
    
  \item Prove 
    \begin{lem}[Foundamental]
      $$
       e:\tau \Rightarrow e\in\lsem\tau\rsem
      $$
    \end{lem}

    \begin{lem}[Adequacy]
      $$
      e\in\lsem\tau\rsem \Rightarrow P(e)
      $$
    \end{lem}
  \end{itemize}
\end{itemize}

\end{frame}

\begin{frame}{Logical Relation}

\begin{itemize}
\item Define $\lsem\tau\rsem$ by $\mathcal{E}\lsem\tau\rsem$ (``good expressions'') and $\mathcal{V}\lsem\tau\rsem$ (``good values'')
  $$
  \begin{array}{lcl}
    \mathcal{V}\lsem\tau_1\to\tau_2\rsem &\defeq& \{e \;\mid\; e \downarrow \tau \wedge \forall v.\;v \in \mathcal{V}\lsem\tau_1\rsem \Rightarrow e\;v\in \mathcal{E}\lsem\tau_2\rsem \} \\
    \mathcal{V}\lsem\tau_1\times\tau_2\rsem &\defeq& \cdots \\
    \mathcal{E}\lsem\tau\rsem &\defeq& \{e \;\mid\; e : \tau \wedge P(e) \wedge \forall v.\;e \Downarrow v \Rightarrow v \in \mathcal{V}\lsem\tau\rsem \}
  \end{array}
  $$
\item $\mathcal{V}\lsem\tau\rsem$ should be defined recusively on $\tau$, but
  $$
  \begin{array}{lcl}
    \mathcal{V}\lsem\mu\alpha.\tau'\rsem &\defeq& \{e \;\mid\; e \downarrow \tau \wedge e \in \mathcal{V}\lsem\textcolor{red}{\tau'[\tau/\alpha]}\rsem \}
  \end{array}
  $$
\end{itemize}

\end{frame}

\begin{frame}{Logical Relation}

\begin{itemize}
\item Solution 1 : Step-index
\item Solution 2 : Postpone substituting for $\alpha$
  $$
  \begin{array}{lcl}
    \mathcal{V}\lsem\alpha\rsem_\rho &\defeq& \rho_2(\alpha) \\
    \mathcal{V}\lsem\mu\alpha.\tau'\rsem_\rho &\defeq& \{e \;\mid\; e \downarrow \rho_1(\tau) \wedge e \in \mathcal{V}\lsem\textcolor{green}{\tau'}\rsem_{\rho[\alpha\mapsto(\tau,\textcolor{red}{?})]} \}
  \end{array}
  $$
\item Solution 2' : Postpone substituting for $\alpha$ + Special logic
  $$
  \begin{array}{lcl}
    \mathcal{V}\lsem\mu\alpha.\tau'\rsem_\rho &\defeq& \mu S.\;\lambda e.\; e \downarrow \rho_1(\tau) \wedge \textcolor{green}{\triangleright} (e \in \mathcal{V}\lsem\textcolor{green}{\tau'}\rsem_{\rho[\alpha\mapsto(\tau,\textcolor{green}{S})]})
  \end{array}
  $$
\end{itemize}

\end{frame}

\begin{frame}{Logical Relation}

\begin{itemize}
\item Encode this special logic in Coq
  \begin{itemize}
    \item HOAS + de Bruijn indices
  \end{itemize}
\item Prove the Foundamental Lemma by induction on typing derivation
  \begin{lem}[Foundamental]
    $$
     e:\tau \Rightarrow e\in\mathcal{E}\lsem\tau\rsem
    $$
  \end{lem}

  \begin{lem}[Bind]
    $$
    \inferrule*
        {E : \tau\rightsquigarrow\tau' \\ e\in \mathcal{E}\lsem\tau\rsem \\ \forall v.\; v\in\mathcal{V}\lsem\tau\rsem \wedge e \leadsto^* v \Rightarrow E[v]\in\mathcal{E}\lsem\tau'\rsem}
        {E[e] \in \mathcal{E}\lsem\tau'\rsem}
    $$
  \end{lem}
\end{itemize}

\end{frame}

\begin{frame}{Currently Working On}

\begin{thm}[Soundness w.r.t. Boundedness]
$$
\begin{array}{l}
\forall f,\tau_1,c,s,\tau_2. \\
\hspace{0.2in} f \downarrow \arrow{(x:\tau_1)}{c(x)}{s(x)}{\tau_2} \Rightarrow \\
\hspace{0.4in} \textcolor{red}{\exists B}. \;\forall v.\; v \downarrow \tau_1 \Rightarrow \\
\hspace{0.4in} \forall e',n.\; f\;v\leadsto^n e' \Rightarrow n\leq B\times c(|v|)
\end{array}
$$
\end{thm}

\begin{itemize}
\item $B$ actually can depend on $v$, when $\tau_1$ is also an arrow type
  \begin{itemize}
  \item For a higher-order function, the constant factor (``width'') can depend on the ``callback''
  \end{itemize}
\end{itemize}

\end{frame}

\end{document}


